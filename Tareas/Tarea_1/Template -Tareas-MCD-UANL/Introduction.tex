\chapter{Introducción}

En el presente trabajo se realiza el análisis estadístico de texto, empleando como fuente a los comentarios de la transmisión en vivo del primer debate presidencial, realizado el día 7 de abril del 2024; los comentarios presentan una estructura lingüística mal estructurada, de igual forma, se identifican Emojis entre apalabras o comentarios completos a partir de los mismos.\\


\section{Primer objetivo}
 Generar información estadística que nos permita visualizar los elementos característicos, que forman parte de los comentarios, como son, la frecuencia palabras, votos a los comentarios (Top 10 de votos), frecuencias de participación en los comentarios (Top 10 de Usuarios) y la relación entre los votos y las replicas.\\

\section{Segundo objetivo}
 Realizar una comparativa entre el numero de menciones y la estadística de intención de voto previo a este debate, a modo de comparación estadística entre la expresión del nombre de los candidatos en los comentarios y la intención de voto documentada hasta ese momento.\\
   


\chapter{descripción de los datos}

Del total de datos unicamente consideraremos el siguiente conjunto de columnas:\\

\begin{itemize}
	\item ID: identificador de comentario
	\item text : Comentario del Usuario.
	\item author : Nombre de usuario que realiza el comentario.
	\item votes : numero de votos recibidos 
	\item replies : número de replicas en los comentarios.\\
\end{itemize} 

Se identifican un total de 1969 comentarios publicados en el video de Youtube (URL: https://www.youtube.com/watch?v=kZaucITWv00\&t=18s), un conteo total de datos de 18958, de los cuales destaca como Top la palabra claudia con un conteo de 419.\\


\begin{table}[H]
	\centering
	\resizebox{\textwidth}{!}{%
		\begin{tabular}{ccccc}
			\rowcolor[HTML]{9B9B9B} 
			\textbf{ID}             & \textbf{text}                                      & \textbf{author}         & \textbf{votes}          & \textbf{replies}        \\
			0                       & El reloj debió verse en todo momento, para ten...  & @celinaramirez6363      & 394                     & 16.0                    \\
			1                       & Me uno a exigir que pongan los relojes de los ...  & @puellacodicum8569      & 555                     & 19.0                    \\
			2                       & Me parece muy bien cómo contestó Maynes, fue d...  & @alondrareal8518        & 79                      & NaN                     \\
			\multicolumn{1}{l}{...} & \multicolumn{1}{l}{...}                            & \multicolumn{1}{l}{...} & \multicolumn{1}{l}{...} & \multicolumn{1}{l}{...} \\
			1967                    & Salud puras citas tras citas así te la llevan ... & @mariacasildaramirezguzman9415       & 0                       & NaN                     \\
			1968                    & Claudia la mejor y vas a ser presidenta el pue...                                 & @tatianavazquez9617  & 0                       & NaN                     \\
			1969                    & Cuánto dinero les pagaste por esos reconocimie...  & @maru641                & 0                       & NaN                    
		\end{tabular}%
	}
	\caption{}
	\label{tab:T1}
\end{table}

A partir del conjunto de datos se generan estadísticos básicos referente a los datos de interés así como representaciones graficas de las distribuciones de los valores que resultan de interés.\\
 
Como primer elemento de revisión se destaca el top 10 de mensajes por usuario, lo que nos indica la participación dentro de los comentarios, identificando los usuarios mas activos e identificar posteriormente su inclinación política con respecto el debate (ver figura \ref{fig:top10}).\\


\begin{figure}[!h]
	\centering
	\includegraphics[width=10cm]{../Datos/Top10Usuarios}
	\caption{Top 10 de usuarios activos con mas comentarios durante el debate.}
	\label{fig:top10}
\end{figure}

\begin{table}[H]
	\centering
	\begin{tabular}{ll}
		\hline
		\multicolumn{2}{c}{\cellcolor[HTML]{000000}{\color[HTML]{FFFFFF} \textbf{df{[}'author'{]}.describe()}}} \\ \hline
		\multicolumn{1}{|l|}{count}        & \multicolumn{1}{l|}{1968}                  \\ \hline
		\multicolumn{1}{|l|}{unique}       & \multicolumn{1}{l|}{1214}                  \\ \hline
		\multicolumn{1}{|l|}{top}          & \multicolumn{1}{l|}{@DoloresMotaHernandez} \\ \hline
		\multicolumn{1}{|l|}{freq}         & \multicolumn{1}{l|}{27}                    \\ \hline
		\multicolumn{1}{|l|}{Name: author} & \multicolumn{1}{l|}{dtype: object}         \\ \hline
	\end{tabular}
\end{table}

% Please add the following required packages to your document preamble:
% \usepackage[table,xcdraw]{xcolor}
% Beamer presentation requires \usepackage{colortbl} instead of \usepackage[table,xcdraw]{xcolor}
\begin{table}[H]
	\centering
	\begin{tabular}{ll}
		\hline
		\multicolumn{2}{c}{\cellcolor[HTML]{000000}{\color[HTML]{FFFFFF} \textbf{df{[}'author'{]}.value\_counts()}}} \\ \hline
		\multicolumn{1}{|l|}{author}                & \multicolumn{1}{l|}{1968} \\ \hline
		\multicolumn{1}{|l|}{@DoloresMotaHernandez} & \multicolumn{1}{l|}{27}   \\ \hline
		\multicolumn{1}{|l|}{@MasterAttitude96}     & \multicolumn{1}{l|}{18}   \\ \hline
		\multicolumn{1}{|l|}{@erendiradelavega9655} & \multicolumn{1}{l|}{18}   \\ \hline
		\multicolumn{1}{|l|}{@psicologia\_unam}     & \multicolumn{1}{l|}{17}   \\ \hline
		\multicolumn{1}{|l|}{@CemeteryGates17}      & \multicolumn{1}{l|}{16}   \\ \hline
		\multicolumn{1}{|l|}{..}                    & \multicolumn{1}{l|}{}     \\ \hline
		\multicolumn{1}{|l|}{@EZ5965}               & \multicolumn{1}{l|}{1}    \\ \hline
		\multicolumn{1}{|l|}{@areval4}              & \multicolumn{1}{l|}{1}    \\ \hline
		\multicolumn{1}{|l|}{@jimmylies}            & \multicolumn{1}{l|}{1}    \\ \hline
		\multicolumn{1}{|l|}{@user-jb2hd4jk3f}      & \multicolumn{1}{l|}{1}    \\ \hline
		\multicolumn{1}{|l|}{@tatianavazquez9617}   & \multicolumn{1}{l|}{1}    \\ \hline
		\multicolumn{1}{|l|}{Name: count, Length: 1214}              & \multicolumn{1}{l|}{dtype: int64}             \\ \hline
	\end{tabular}
\end{table}

Al realizar la estadística correspondiente a los votos, valor que corresponde a likes en los comentarios, podemos observar que un porcentaje alto de los comentarios no presentan votos, mientras que un porcentaje alto de votos se distribuye entre pocos comentarios, pero no el mayor, en la figura \ref{fig:top10V} podemos identificar comentarios individuales con un alto numero de votos, los cuales son poco frecuentes pero muy significativos.\\

\begin{figure}[!h]
	\centering
	\includegraphics[width=10cm]{../Datos/Top10Votos}
	\caption{Top 10 del mayor numero frecuencia de votos en los comentarios.}
	\label{fig:top10V}
\end{figure}


\begin{table}[H]
	\centering
	\begin{tabular}{ll}
		\hline
		\multicolumn{2}{c}{\cellcolor[HTML]{000000}{\color[HTML]{FFFFFF} \textbf{df{[}'votes'{]}.describe()}}} \\ \hline
		\multicolumn{1}{|l|}{count}  & \multicolumn{1}{l|}{170} \\ \hline
		\multicolumn{1}{|l|}{unique} & \multicolumn{1}{l|}{130} \\ \hline
		\multicolumn{1}{|l|}{top}    & \multicolumn{1}{l|}{0}   \\ \hline
		\multicolumn{1}{|l|}{freq}   & \multicolumn{1}{l|}{949} \\ \hline
		\multicolumn{1}{|l|}{Name: votes}                 & \multicolumn{1}{l|}{dtype: object}                 \\ \hline
	\end{tabular}
\end{table}

De forma general podemos ver que la distribución de los valores tiene una distribución exponencial, los valores, de igual forma se observa que aquellos con mayor numero de votos presentan una muy baja frecuencia siendo mas los que presentan pocos votos.\\ 

% Please add the following required packages to your document preamble:
% \usepackage[table,xcdraw]{xcolor}
% Beamer presentation requires \usepackage{colortbl} instead of \usepackage[table,xcdraw]{xcolor}
\begin{table}[H]
	\centering
	\begin{tabular}{ll}
		\hline
		\multicolumn{2}{c}{\cellcolor[HTML]{000000}{\color[HTML]{FFFFFF} \textbf{df{[}'votes'{]}.value\_counts()}}} \\ \hline
		\multicolumn{1}{|l|}{0}   & \multicolumn{1}{l|}{949} \\ \hline
		\multicolumn{1}{|l|}{1}   & \multicolumn{1}{l|}{260} \\ \hline
		\multicolumn{1}{|l|}{2}   & \multicolumn{1}{l|}{123} \\ \hline
		\multicolumn{1}{|l|}{3}   & \multicolumn{1}{l|}{71}  \\ \hline
		\multicolumn{1}{|l|}{4}   & \multicolumn{1}{l|}{60}  \\ \hline
		\multicolumn{1}{|l|}{...} & \multicolumn{1}{l|}{16}  \\ \hline
		\multicolumn{1}{|l|}{204} & \multicolumn{1}{l|}{1}   \\ \hline
		\multicolumn{1}{|l|}{112} & \multicolumn{1}{l|}{1}   \\ \hline
		\multicolumn{1}{|l|}{332} & \multicolumn{1}{l|}{1}   \\ \hline
		\multicolumn{1}{|l|}{567} & \multicolumn{1}{l|}{1}   \\ \hline
		\multicolumn{1}{|l|}{80}  & \multicolumn{1}{l|}{1}   \\ \hline
		\multicolumn{1}{|l|}{Name: count, Length: 130,}             & \multicolumn{1}{l|}{dtype: int64}             \\ \hline
	\end{tabular}
\end{table}

En el comportamiento de las replicas podemos identificar una mayor frecuencia en un menor numero de replicas por comentario, es decir la gran mayoría de pocas replicas ocurren mas frecuentemente, al igual que la grafica anterior, podemos observar un comportamiento mas general en la figura (FALTA), donde se identifican la poca frecuencia de replicas continuas en en comentarios, es decir solo algunos comentarios son significativos y acumulan un alto numero de replicas.\\

\begin{figure}[!h]
	\centering
	\includegraphics[width=10cm]{../Datos/Top10Replicas}
	\caption{Top 10 del mayor numero de frecuencias de replicas en los comentarios.}
	\label{fig:top10R}
\end{figure}


Al observar la distribución de votos Vs replicas de los comentarios, se puede identificar una relación entre los comentarios con mayor numero de votos y los comentarios con mayor numero de replicas, en ambos conjuntos de datos se observa una distribución de tipo exponencial, teniendo los numero altos de participaciones y replicas una probabilidad significativamente menor siendo mayor las replicas a los votos (ver figura \ref{fig:FyD}).\\


\begin{figure}[!h]
	\centering
	\includegraphics[width=13cm]{../Datos/AcumulacionYdistribuciones}
	\caption{grafico de frecuencias y distribución}
	\label{fig:FyD}
\end{figure}

\chapter{Análisis de los comentarios}





\chapter{Conclusiones}

Según las encuestas y análisis publicados antes del primer debate presidencial en México 2024, las intenciones de voto se distribuían de la siguiente manera:

Claudia Sheinbaum (Morena, PT y PVEM): 49% (según encuesta “flash” publicada el 7 de abril de 2024)
Xóchitl Gálvez (PRI-PAN-PRD): 26% (según encuesta “flash” publicada el 7 de abril de 2024)
Jorge Álvarez Máynez (Movimiento Ciudadano): 18% (según encuesta “flash” publicada el 7 de abril de 2024)


La encuesta de El Financiero del 1 de abril de 2024 mostraba que Sheinbaum lideraba con un 35%, seguida de Gálvez con un 25% y Máynez con un 15%.

La encuesta de FactoMétrica y Reporte Índigo publicada el 9 de abril de 2024 mostraba que Sheinbaum lideraba con un 69\%, seguida de Gálvez con un 26,5\% y Máynez con un 4,5\%.

La encuesta “flash” publicada el 7 de abril de 2024 mostraba que Sheinbaum lideraba con un 49\%, seguida de Gálvez con un 26\% y Máynez con un 18\%.